
\فصل{مقدمه}

\قسمت{تعریف مسئله}

معماری میکروسرویس سبکی از معماری است که در آن یک برنامه‌ی کاربردی در قالب مجموعه‌ای از اجزا یا ماژول‌ها ساخته می شود. هر ماژول از یک کار یا هدف تجاری خاص پشتیبانی می‌کند و از یک واسط که به صورت کامل تعریف شده و ساده است مانند واسط برنامه نویسی کاربردی\پاورقی{Application Programming Interface (API)} برای برقراری ارتباط با دسته‌های دیگر از برنامه‌ها یا خدمات استفاده می‌کند\مرجع{buildingmicroservicesbook}.

در سال های اخیر، میکروسرویس ها محبوبیت بیشتری پیدا کرده اند و شرکت‌ها ترجیح می‌دهند از معماری یکپارچه به معماری میکروسرویس حرکت کنند. به طور معمول، ممکن است صدها تا هزاران میکروسرویس در سیستم‌های مبتنی بر معماری میکروسرویس در برنامه‌ها با مقیاس بزرگ گنجانده شوند. به عنوان مثال، بیش از 500 میکروسرویس نتفلیکس\پاورقی{Netflix} روزانه تقریباً 2 میلیارد درخواست واسط برنامه‌نویسی کاربردی را انجام می دهند. \مرجع{netflixsample2015}، در حالی که سیستم وی‌چت تنسنت\پاورقی{WeChat Tencent} از بیش از 3000 سرویس تشکیل شده است که بر روی بیش از 20000 ماشین اجرا می‌شود\مرجع{Zhou2018}. 

سبک معماری میکروسرویس از معماری بر محور خدمت\پاورقی{Service Oriented Architecture (SOA)} پدید آمده است و تاکید آن نیز بر ایجاد خدمات ریزدانه‌، توسعه-عملیات\پاورقی{Development-Operations (DevOps)} و روش‌های چابک\پاورقی{Agile} است\مرجع{waseem2020}. 
دو سبک معماری میکروسرویس و معماری بر محور خدمت، به طور قابل توجهی از نظر ویژگی‌های هر خدمت، مانند ارتباطات خدمات و به اشتراک گذاری مولفه\پاورقی{Component}ها متفاوت هستند؛ به عنوان مثال، معماری میکروسرویس می‌تواند شامل صدها سرویس ریزدانه باشد که با هم یک خدمت را ارائه می‌کنند، در حالی ‌که هر کدام از این میکروسرویس‌ها با سرعت زیاد تکامل می یابند. اما سبک معماری بر محور خدمت شامل خدمات درشت‌دانه‌تر و پایدارتری است. در معماری میکروسرویس، ارتباط از طریق یک لایه واسط برنامه‌نویسی کاربردی برقرار می‌شود، در حالی که در معماری بر محور خدمت می‌توان از طریق یک واسط برنامه‌نویسی کاربردی (API)  در کنار یک گذرگاه خدمات سازمانی\پاورقی{Enterprise Service Bus (ESB)} ارتباطات را برقرار کرد. در کل معماری میکروسرویس مقیاس‌پذیری، جداسازی و کنترل بهتری در فرآروند پیاده‌سازی برنامه (به عنوان مثال در مراحل توسعه، آزمون، استقرار) نسبت به معماری بر محور خدمت ارائه می‌کند\مرجع{msavssoa2016}.
این تفاوت‌های اساسی چالش‌هایی را در آزمون راهبردهای خاص برنامه‌های مبتنی بر معماری میکروسرویس ایجاد می‌کند. هدف از ایجاد آزمون‌ها یافتن احتمال خطاهای پیاده‌سازی یا خطاهایی است که به کاهش کیفیت منجر می‌شوند و یا خطاهایی است که قابلیت استفاده یک برنامه یا سیستم را کاهش می‌دهند. 


آزمون نرم‌افزار از مجموعه تصدیق‌کننده‌های پویا تشکیل شده است که ارزیابی می کنند، آیا یک سیستم، رفتار مورد انتظار را در مجموعه‌ی محدودی از موارد آزمون (که به طور مناسب از دامنه‌ی اجرای معمولاً نامحدود انتخاب شده‌اند) ارائه می‌دهد؟\مرجع {SWEBOK2014}. آزمون‌ها مجموعه‌ای از فعالیت‌ها هستند که هدفشان یافتن خطاهای احتمالی در طراحی، پیاده‌سازی، کیفیت یا قابلیت استفاده یک برنامه یا سیستم است. تعداد سرویس‌ها، فرآیندهای ارتباطی، وابستگی‌ها، نمونه‌ها، ارتباطات شبکه و سایر متغیرها بر روش‌های آزمون برنامه‌ها با معماری میکروسرویس تأثیر می‌گذارند. این نوع از برنامه‌ها به دلیل ماهیت پیچیده و رفتار پویای خود، چالش های مهمی را برای آزمون ایجاد می کنند. برای رسیدن به هدفی که گفته شد، در طول فرآروند آزمون، باید رفتارهای همزمان میکروسرویس‌های مختلف و تعاملات بین آنها درک شود\مرجع{Beschastnikh2016}. 

\قسمت{اهمیت موضوع}

تشخیص عیوب و اشکالات یک نرم‌افزار در مرحله طراحی، یک گام مهم در مهندسی نرم‌افزار است. اهمیت این فرآیند از این واقعیت ناشی می شود که به بالا رفتن کیفیت عملکرد نرم افزار کمک می کند و باعث اطمینان بیشتر و کارآمدتر شدن آن در خدمت به کاربران می شود. با تشخیص فعالانه نقص‌ها در مرحله طراحی، توسعه‌دهندگان می‌توانند از مشکلات بی‌شماری پایین‌دستی که در غیر این صورت پس از استقرار نرم‌افزار و استفاده توسط کاربران نهایی ایجاد می‌شوند، جلوگیری کنند. این فرآیند به‌ویژه در پروژه‌های مقیاس بزرگ، که در آن‌ها حتی اشکالات یا نقص‌های جزئی می‌تواند فاجعه‌بار باشد و منجر به خسارات مالی قابل‌توجه و آسیب به اعتبار سازمان شود، اهمیت دارد.

مزایای مالی تشخیص نقص در مرحله طراحی قابل توجه است و به ویژه برای پروژه های بزرگ جذاب است. با شناسایی و رفع خطاها در مراحل اولیه طراحی، سازمان ها می توانند خطرات ناشی از خطاها و تاخیرهای پرهزینه را به حداقل برسانند، در منابع با ارزش صرفه جویی کنند و کارایی کلی پروژه را بهبود بخشند. این رویکرد همچنین با اجازه دادن به مدیران پروژه برای برآورد نیازهای منابع و تخصیص موثرتر منابع، فرآیند توسعه را بهبود می بخشد. در مجموع، تشخیص عیوب و اشکالات در مرحله طراحی برای توسعه و استقرار موفق نرم‌افزار ضروری است و می‌تواند تأثیر قابل‌توجهی بر بازگشت سرمایه برای سازمان‌ها داشته باشد.

چالش‌هایی بر سر راه آزمون برنامه‌ها با معماری پیچیده و رفتار پویای میکروسرویس‌ها وجود دارد، در پژوهش‌های پیشین سعی شده است که برای چالش‌های ایجاد شده در این مسیر روش‌ها و رویکردهای جدیدی ارائه شود، در این روش‌ها تلاش شده است تا با وجود مولفه‌هایی که به طور مستقل از هم قابل استقرار هستند و در عین حال روابط پیچیده‌ای دارند، درستی عملکرد کلی سامانه سنجیده شود. یکی از رویکردهای آزمون معرفی شده رویکرد مدل‌رانه است، پژوهشگران در این رویکرد سعی کرده‌اند با روش‌های مدل‌سازی و ارائه‌ی توصیف صوری از میکروسرویس‌ها صحت عملکرد و نحوه تعامل میکروسرویس‌ها راصحت‌سنجی کنند، با وجود روش‌های پیشنهادی جدید، آزمون سامانه‌ها با معماری میکروسرویس در حوزه صنعت همچنان با چالش‌های زیادی روبه‌رو است. \مرجع{waseem2020}


\قسمت{اهداف پژوهش}

هدف ما در این پایان‌نامه ارائه‌ی روشی مبتنی بر آزمون مدل-رانه برای آزمون برنامه‌ها و سیستم‌ها با معماری میکروسرویس بوده‌ است. روش آزمونی که در این پایان‌نامه ارائه دادیم، موارد آزمون را طبق یک معیار پوشش بر اساس منطق، به صورت خودکار استخراج می‌کند و همچنین با استفاده از یک ابزار آماده به کار برای تحلیل توصیفات صوری آن‌ها را در محیطی محدود بر روی توصیف صوری برنامه اجرا می‌کند.

برای تولید موارد آزمون از معماری برنامه به صورت خودکار، ابتدا مدل‌های گردش کار برنامه که به زبان یاول\مرجع{yawlbook} تولید شده‌اند، به ابزاری که تولید کردیم، ورودی داده می‌شود سپس ما با تحلیل مدل ورودی آن را با روشی که توسعه داده‌شده‌ی روش پیشنهادی ریواده و همکاران است\مرجع{rivadeh2022}، آن را به توصیفی صوری به زبان الوی\مرجع{alloybook} تبدیل می‌کنیم و سپس موارد آزمون را از مکان‌هایی از برنامه که جریان کنترل برنامه را تعیین می‌کنند استخراج می‌کنیم و بر اساس مسند\پاورقی{predicate}های موجود در آن‌ها موارد آزمون را به صورت خودکار طبق پوشش گزاره‌ی فعال بند محدود\پاورقی{Restricted Active Clause Coverage (RACC)} تولید می‌کنیم.

در این پایان‌نامه سعی کردیم ترجمه‌ی مدل‌ها از یاول به الوی را به صورت خودکار، در ابزار موجود برای طراحی مدل‌های گردش کاری در یاول، انجام دهیم؛ با استفاده از ترجمه‌ی انجام‌شده تحلیل‌هایی بر روی ساختار میکروسرویس‌ها انجام می‌دهیم و نتایج را در ابزار نمایش می‌دهیم؛ علاوه بر این‌ها تولید موارد آزمون را نیز طبق روشی که ارائه دادیم به صورت خودکار و با تولید همه‌ی مقادیر ممکن از روی مسندهای توصیف صوری بر اساس معیار RACC ایجاد می‌کنیم و در همان ابزار گفته شده به آزمون‌کننده‌ نمایش می‌دهیم. علاوه بر این‌ها موارد آزمون تولید‌شده بر روی توصیف صوری اعمال می‌شود و نتایج آزمون را نیز در ابزار نشان می‌دهیم.


\قسمت{ساختار پایان‌نامه}

ساختار پایان‌نامه در ادامه به شرح زیر است: بخش دوم شامل مفاهیم اولیه‌ای است که برای درک روش پیشنهادی و کارهای مرتبط با این پایان‌نامه نیاز هستند. در بخش سوم کارهای پیشین مرتبط با موضوع این پایان‌نامه را بررسی کردیم و سعی کردیم آن‌ها را بر اساس فراوانی موضوعات اصلی تحقیقی رایج در موضوعات پنج‌گانه دسته‌بندی کنیم. در بخش چهارم روش ارائه‌شده‌ی خود را با استفاده از یک نمونه‌ی برگرفته شده از یک برنامه‌ی واقعی شرح داده‌ایم. و در بخش پایانی نیز به ارزیابی روش پرداختیم. 
