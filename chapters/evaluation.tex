
\فصل{ارزیابی}

\قسمت {معیارهای ارزیابی}
در این بخش به ارزیابی روش ارائه شده در 
 \autoref{فصل:روش پیشنهادی}
می‌پردازیم ما برای ارزیابی روش ارائه‌‌شده‌ی خود از آزمون جهش استفاده کردیم. قصد ما ارزیابی آزمون‌هایی ست که از توصیفات صوری استخراج شده‌اند،  همچنین همان‌طور که در روش ارائه‌شده بیان‌شد، موارد آزمون بر اساس معیار \لر{RACC} از روی مسند‌های موجود در جریان‌های خروجی وظایف گردش کار به دست می‌آیند. در نتیجه سعی کردیم با انتخاب عملگر‌های جهش مناسب، آن‌ها را بر روی گزاره‌های تشکیل‌دهنده‌ی مسند‌ها اعمال کنیم و موارد آزمون استخراج‌شده را از این نظر که چقدر توانایی کشتن جهش‌یافته‌ها را دارند، ارزیابی می‌کنیم.

در ارزیابی، از دو برنامه با اندازه‌های متفاوت استفاده کرده‌ایم اما از عملگر‌های جهش مشابه برای تغییر در آن‌ها استفاده کرده‌ایم. 
دو برنامه‌ای که برای ارزیابی استفاده شده‌اند به ترتیب دارای ۵ و ۱۰ میکروسرویس هستند و بر روی هر کدام، ۴ معیار جهش اعمال شده‌ است. عملگرهای جهش انتخابی، عبارتند از، جایگزینی عملگرهای رابطه‌ای، جایگزینی عملگرهای شرطی، حذف عملگرهای یگانی\پاورقی{unary} و جایگزینی متغیرهای عددی\پاورقی{scalar} 


\قسمت{قسمت ارزیابی برنامه‌ی ارائه‌ی محتوای چندرسانه‌ای}
ما برای ارزیابی روش خود، یک برنامه‌ی ارائه‌ی محتوای چندرسانه‌ای را طراحی کردیم، این برنامه نیز مشابه با برنامه‌ی خرید برخط، برگرفته از یک برنامه‌ی واقعی، ایجاد شده است. \مرجع{cinama2023} برنامه‎ی ارائه‌ی محتوای چندرسانه‌ای در دنیای واقع است. 

برنامه‌ای که طراحی کرده‌ایم شامل 5 میکروسرویس است که طبق شکل ~\رجوع{شکل: گردش کاری یک برنامه‌ی نمونه‌ی ارائه محتوای چندرسانه‌ای} برای ارائه‌ی خدمت نهایی، یک گردش کاری را تشکیل دادند. همان طور که در شکل مشخص است، برنامه دارای دو نوع سرویس تماشای برخط\پاورقی{streaming} و بارگیری\پاورقی{download} است. برنامه پس از استفاده‌ی کاربر از هر یک از این سرویس‌ها، کاربر را به اندازه‌ی قیمت محصول ارائه‌شده شارژ می‌کند. اگر موجودی حساب کاربر به اندازه‌ی کافی باشد، مبلغ محصول از حساب کاربر کسز می‎شود و در غیر این صورت، سرویس پرداخت کاربر را به درگاه بانکی می‌برد تا کاربر حساب خود را افزایش اعتبار بدهد، و در نهایت نیز خدمت برنامه به اتمام می‌رسد. 

در برنامه‌ی طراحی‌شده، ۵ سرویس تماشای برخط، بارگیری، مدیریت پرداخت تماشای برخط، مدیریت پرداخت بارگیری و پرداخت وجود دارند.

شکل~\رجوع{شکل: گردش کاری یک برنامه‌ی نمونه‌ی ارائه محتوای چندرسانه‌ای} نشان‌دهنده‌ی گردش کاری این برنامه است.
\شروع{شکل}[t]
\centerimg{multimedia-app}{10cm}
\vspace{0.5em}
\شرح{گردش کاری یک برنامه‌ی نمونه‌ی ارائه محتوای چندرسانه‌ای}
\برچسب{شکل: گردش کاری یک برنامه‌ی نمونه‌ی ارائه محتوای چندرسانه‌ای}
\پایان{شکل}

عملگرهای جهشی که برای ارزیابی موارد آزمون‌ استخراج‌شده به کار رفته‌اند، جایگزینی عملگرهای رابطه‌ای، جایگزینی عملگرهای شرطی، حذف عملگرهای یگانی\پاورقی{unary} و جایگزینی متغیرهای عددی هستند. این عملگرهای جهش بر روی مسندهای شرطی نوشته‌شده در خروجی وظایف‌ مدیریت پرداخت تماشای برخط و مدیریت پرداخت بارگیری اِعمال شده‌اند، در جدول ~\رجوع{جدول:جهش‌یافته‌های برنامه‌ی ارائه‌ی محتوای چندرسانه‌ای} مسند شرطی استفاده شده برای تولید جهش‌یافته، نمونه مسند شرطی جهش‌یافته و عملگر جهش استفاده‌شده را نمایش می‌دهد. این مسند‌ها در میکروسرویس‌های مدیریت پرداخت تماشای برخط و مدیریت پرداخت بارگیری استفاده شده‌اند.

\شروع{لوح}[t]
\وسط‌چین
\شرح{جهش‌یافته‌های برنامه‌ی ارائه‌ی محتوای چندرسانه‌ای}
\برچسب{جدول:جهش‌یافته‌های برنامه‌ی ارائه‌ی محتوای چندرسانه‌ای}


\شروع{جدول}{|p{0.45\linewidth} | p{0.45\linewidth} | p{0.1\linewidth}|}

\خط‌پر
  \toprule
مسند شرطی اصلی & نمونه مسند جهش‌یافته &  عملگر جهش   \\
  \midrule
 $((fee > customer\_credit\, \&\&\, service\_type = "stream")\, ||\, is\_free = 1)$ &  $((fee < customer\_credit\, \&\&\, service\_type = "stream")\, ||\, is\_free = 1)$  &  ROR \\
 	&  $((fee > customer\_credit\, \&\&\, service\_type = "stream"\, \&\&\, is\_free = 1)$  &  COR \\
	&  $((fee > customer\_credit \,\&\&\, service\_type = "stream")\, ||\, \textasciitilde\, is\_free = 1)$  &  UOD \\
	&  $((fee > customer\_credit\, \&\&\, service\_type = "stream") \,||\, customer\_credit = 1)$  &  SVR \\
 \خط‌پر
 $((fee > customer\_credit\, \&\&\, service\_type = "download")\, ||\, is\_free = 1)$ &  $((fee > customer\_credit\, \&\&\, service\_type = "download")\, ||\, is\_free > 1)$  &  ROR \\
 	& $((fee > customer\_credit\, ||\, service\_type = "download"\, ||\, is\_free = 1)$  &  COR \\
 	& $((fee > customer\_credit \,\&\&\, \textasciitilde\, (service\_type = "download"))\, ||\, is\_free = 1)$  &  UOD \\
	&  $((fee > is\_free\, \&\&\, service\_type = "download") \,||\, is\_free = 1)$  &  SVR \\
 \خط‌پر
 
\پایان{جدول}
\پایان{لوح}

با اعمال عملگرهای جهش بر روی مسند‌های شرطی و تکرار اجرای آزمون‌ها $۹.۳۷۵$ درصد از آزمون‌ها به شکست انجامید. در جدول ~\رجوع{جدول:درصد آزمون‌های شکست‌خورده بعد از تولید جهش‌یافته‌ها در برنامه‌ی ارائه‌ی محتوای چندرسانه‌ای} به تفکیک عملگرها درصد شکست آزمون‌ها بعد از تولید جهش‌یافته‌ها و اجرای دوباره‌ی آزمون‌ها آمده است. می‌توان گفت روش ما کارایی بالایی دارد و فقط در عملگر حذف عملگرهای یگانی کارایی خوبی ندارد و باعث شکست خوردن موارد آزمون‌ استخراج‌شده نشد‌ه‌است. 



\vspace{1.5em}

\شروع{لوح}[h]
\تنظیم‌ازوسط
\شرح{درصد آزمون‌های شکست‌خورده بعد از تولید جهش‌یافته‌ها در برنامه‌ی ارائه‌ی محتوای چندرسانه‌ای}

\شروع{جدول}{|c|c|}
\خط‌پر 
عملگر جهش & درصد آزمون‌های شکست‌خورده بعد از تولید جهش‌یافته‌ها   \\
\خط‌پر \خط‌پر 
ROR & $18.75$ \\
\خط‌پر
COR & $12.5$ \\
\خط‌پر
UOD & $0$ \\
\خط‌پر
SVR & $12.5$ \\
\خط‌پر
\پایان{جدول}

\برچسب{جدول:درصد آزمون‌های شکست‌خورده بعد از تولید جهش‌یافته‌ها در برنامه‌ی ارائه‌ی محتوای چندرسانه‌ای}
\پایان{لوح}




\قسمت{قسمت ارزیابی برنامه‌ی خرید برخط}
برنامه‌ای که در این قسمت روش ارائه‌ی شده‌ی خود را بر روی آن ارزیابی می‌کنیم همان برنامه‌ای است که در فصل
 \autoref{فصل:روش پیشنهادی}
 گفتیم است. 

برنامه‌ای که گفته‌شد برنامه‎ی خرید برخط است و شامل میکروسرویس‌های بررسی‌کننده‌ی امنیتی درخواست‌ها، ورود با استفاده از گوگل، ورود با استفاده از توئیتر، ورود با استفاده از لینکدین، سرویس خرید (انتخاب موارد خرید)، پرداخت با کارت اعتباری، پرداخت با پی‌پل و تایید پرداخت است.
شکل~\رجوع{شکل: گردش کاری یک برنامه‌ی نمونه برای خرید برخط}  نشان‌دهنده‌ی گردش کاری این برنامه است.

عملگرهای جهشی که برای ارزیابی موارد آزمون‌ استخراج‌شده به کار رفته‌اند، جایگزینی عملگرهای رابطه‌ای، جایگزینی عملگرهای شرطی، حذف عملگرهای یگانی و جایگزینی متغیرهای عددی هستند. این عملگرهای جهش بر روی مسندهای شرطی نوشته‌شده در خروجی وظایف‌ انتخاب روش ورود و بررسی‌کننده‌ی امنیتی اِعمال شده‌اند، در جدول ~\رجوع{جدول:جهش‌یافته‌های برنامه‌ی خرید برخط} مسند شرطی استفاده شده برای تولید جهش‌یافته، نمونه مسند شرطی جهش‌یافته و عملگر جهش استفاده‌شده را نمایش می‌دهد.

\شروع{لوح}[t]
\وسط‌چین
\شرح{جهش‌یافته‌های برنامه‌ی خرید برخط}
\برچسب{جدول:جهش‌یافته‌های برنامه‌ی خرید برخط}


\شروع{جدول}{|p{0.45\linewidth} | p{0.45\linewidth} | p{0.1\linewidth}|}

\خط‌پر
  \toprule
مسند شرطی اصلی & نمونه مسند جهش‌یافته &  عملگر جهش   \\
  \midrule
 $(((login\_by\_linkedin = 1)  \,\&\&\, (login\_by\_twitter = 0))  \,\&\&\, (login\_by\_google = 0))$
  & $(((login\_by\_linkedin > 1)  \,\&\&\, (login\_by\_twitter = 0)) \,\&\&\,and (login\_by\_google = 0))$
  &  ROR \\
	  & $(((login\_by\_linkedin = 1) \, ||\,  (login\_by\_twitter = 0))  \,\&\&\, (login\_by\_google = 0))$ 
  &  COR \\
	  &   $(((login\_by\_linkedin = 1)  \,\&\&\, (login\_by\_twitter = 0))  \,\&\&\, (login\_by\_google \neq 0))$
  &  UOD \\
 
	  &   $(((login\_by\_linkedin = 1)  \,\&\&\, (login\_by\_twitter = 0))  \,\&\&\, (login\_by\_linkedin = 0))$
  &  SVR \\
 \خط‌پر
 
\پایان{جدول}

\پایان{لوح}

با اعمال عملگرهای جهش بر روی مسند‌های شرطی و تکرار اجرای آزمون‌ها $15.82$ درصد از آزمون‌ها به شکست انجامید. در جدول ~\رجوع{جدول:درصد آزمون‌های شکست‌خورده بعد از تولید جهش‌یافته‌ها در برنامه‌ی خرید برخط} به تفکیک عملگرها درصد شکست آزمون‌ها بعد از تولید جهش‌یافته‌ها و اجرای دوباره‌ی آزمون‌ها آمده است. می‌توان گفت روش ما در این ارزیابی هم کارایی بالایی دارد.


\شروع{لوح}[h]
\تنظیم‌ازوسط
\شرح{درصد آزمون‌های شکست‌خورده بعد از تولید جهش‌یافته‌ها در برنامه‌ی خرید برخط}

\شروع{جدول}{|c|c|}
\خط‌پر 
عملگر جهش & درصد آزمون‌های شکست‌خورده بعد از تولید جهش‌یافته‌ها   \\
\خط‌پر \خط‌پر 
ROR & $20$ \\
\خط‌پر
COR & $13.3$ \\
\خط‌پر
UOD & $6.67$ \\
\خط‌پر
SVR & $23.3$ \\
\خط‌پر
\پایان{جدول}

\برچسب{جدول:درصد آزمون‌های شکست‌خورده بعد از تولید جهش‌یافته‌ها در برنامه‌ی خرید برخط}
\پایان{لوح}



