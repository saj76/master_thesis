
\فصل{کارهای پیشین}

در این فصل از پایان‌نامه، کارهای پیشین انجام‌شده روی مسئله به تفصیل توضیح داده می‌شود.


\قسمت{پژوهش‌های مربوط به آزمون میکروسرویس‌ها}

در راستای آزمون برنامه‌ها با معماری میکروسرویس‌ روش‌های گوناگونی ارائه‌ شده‌اند. می‌توان اغلب روش‌های ارائه‌شده را در پنج زمینه‌\پاورقی{theme}ی اصلی دسته‌بندی کرد به طوری که آن‌ها مطالعات اصلی در راستای آزمون نرم‌افزارها با معماری میکروسرویس را مشخص می‌کنند، این 5 دسته، شامل آزمون خودکار\پاورقی{automated testing}، معماری ، توسعه-عملیات و  یکپارچه‌سازی مستمر\پاورقی{continuous integration}، عملکرد\پاورقی{performance} و آزمون مبتنی بر مدل\پاورقی{model-based}. البته این نکته قابل ذکر است که یک مطالعه ممکن است چندین موضوع از این 5 زمینه‌ی گفته‌شده را در بر بگیرد.

\زیرقسمت{آزمون خودکار}
این زمینه، مطالعات عمده‎ای را پوشش می‌دهد که آزمون‌های خاصی را در قالب آزمون خودکار مورد بحث قرار می‌دهند. به عنوان مثال، کوئنوم\زیرنویس{Quenum} و آکنین\زیرنویس{Aknine} یک رویکرد آزمون خودکار مبتنی بر توصیف صوری و عوامل هوشمند (به عنوان مثال، آزمون خودکارشده‌ی \لر{LASTA}) برای استخراج موارد آزمون (به عنوان مثال، آزمون‌های واحد و پذیرش) برای برنامه‌ها با معماری میکروسرویس ارائه می‌کنند \مرجع{quenum2018}. 

به طور مشابه، شنگ\زیرنویس{Shang} و همکاران. یک طرح مبتنی بر گراف و سناریو-رانه را برای تحلیل، آزمون و استفاده مجدد از میکروسرویس‌ها معرفی کردند\مرجع{ma2018}. این رویکرد از ابزار \لر{swagger} برای استخراج خودکار فراخوانی‌ها بین میکروسرویس‌ها استفاده می‌کند و وابستگی‌های بین میکروسرویس‎ها به صورت خودکار به‌روز می‌شوند. در نتیجه امکان بازیابی خودکار موارد آزمون مورد نیاز در مواجهه با تغییرات میکروسرویس را فراهم می‌کند. 

\زیرقسمت{معماری}
در این زمینه‌ی تحقیق، بحث عمده بر استفاده یا در نظر گرفتن مصنوعات معماری (به عنوان مثال، اجزای معماری، ویژگی‌های کیفیت) برای آزمون برنامه‌های مبتنی بر میکروسرویس متمرکز است. به طور دقیق‌تر، مطالعات اصلی، رویکردهای آزمونی را ارائه می‌کنند که از اجزای طراحی میکروسرویس‌ها (به عنوان مثال، سرویس‌ها و رابط سرویس)، راهبردهای تجزیه‌ی برنامه‌ها، و روش‌های ارتباطی میکروسرویس‌ها (مانند پروتکل‌های همگام\پاورقی{synchronous} یا ناهمگام\پاورقی{asynchronous}) برای ایجاد موارد آزمون برای آزمون برنامه‌های کاربردی مبتنی بر معماری میکروسرویس استفاده می‌کنند. علاوه بر این، چندین مطالعه، چگونگی آزمون و ارزیابی قابلیت اطمینان، انعطاف‌پذیری، و معماری برنامه‌های کاربردی مبتنی بر معماری میکروسرویس را در این زمینه می‌گویند.

به عنوان مثال هورهیادی\زیرنویس{Heorhiadi} و همکاران روش گرملین\زیرنویس{Gremlin} را با تمرکز بر شبکه‌های ارتباطی بین میکروسرویس‌ها ارائه داده‌اند. گرملین چارچوبی برای آزمون سیستماتیک قابلیت‌های مواجهه با  شکست در میکروسرویس‌هاست. گرملین بر اساس این اصل است که میکروسرویس‌ها به طور معمول جفت‌شدگی کمی دارند و در عوض بر الگوهای استاندارد تبادل پیام در شبکه متکی هستند. گرملین به آزمون‌کننده اجازه می‌دهد تا به راحتی آزمون‌ها را طراحی کند و آن‌ها را با دستکاری پیام‌های بین سرویس‌ها در لایه شبکه اجرا کند. این پژوهش نشان می‎دهد که می‌توان از گرملین برای بیان سناریوهای شکست رایج به نحوی استفاده کرد تا توسعه‌دهندگان یک برنامه قادر باشند اشکالات ناشناخته‌ی قبلی در کد مدیریت شکست خود را بدون تغییر برنامه، شناسایی کنند\مرجع{heorhiadi2016}.

همچنین در پژوهشی که لوتز\زیرنویس{Lotz} و همکارانش ارائه داده‌اند یک مطالعه موردی را بررسی می‌کنند که امکان‌سنجی و اثرات احتمالی تغییر معماری نرم‌افزار کمک‌کننده به راننده را به یک معماری میکروسرویس شامل می‌شود. همچنین آزمون سیستم را برای این مورد مطالعه انجام می‌دهند. نتایج نشان می‌دهد که معماری‌ میکروسرویس می‌تواند پیچیدگی و مراحل فرآروند زمان‌بر را کاهش دهد و سیستم‌های نرم‌افزار خودرو را برای چالش‌های آتی آماده کند تا زمانی که اصول معماری‌ میکروسرویس به دقت دنبال شوند\مرجع{lotz2019}. 

\زیرقسمت{توسعه-عملیات و  یکپارچه‌سازی مستمر}
توسعه-عملیات و یکپارچه‌سازی مستمر، دارای طیف وسیعی از اقدامات است (به عنوان مثال، یکپارچه سازی مستمر\پاورقی{continuous integration (CI)}، تحویل مستمر\پاورقی{continuous delivery (CD)}، آزمون، استقرار) آن‌ها با هدف ارائه‌ی سیستم‌های نرم‌افزاری قابل اعتماد با تشویق همکاری نزدیک بین کارکنان توسعه و عملیات پیشنهاد شده‌اند. این زمینه‌ی پژوهشی شامل مطالعاتی است که رویکردهای آزمون و ابزارهای مورد استفاده برای برنامه‌های مبتنی بر معماری میکروسرویس در توسعه-عملیات و یکپارچه‌سازی مستمر را گزارش می‌کنند. خودکارسازی آزمون یک عامل کلیدی در موفقیت با توسعه-عملیات است. به عنوان مثال، کارگر و حنیفی زاده یک روش خودکار را برای پشتیبانی از تست رگرسیون میکروسرویس‌ها در تحویل مداوم پیشنهاد کردند\مرجع{kargar2018}.

مارسل\زیرنویس{Marcel} و کریستوس گرابمن\زیرنویس{Christos Grobmann} برنامه‌های میکروسرویسی را برای سیستم عامل شبکه باز (\لر{ONOS}) با ایجاد محیط یکپارچه‌سازی مستمر توسعه دادند و آزمون کردند. آن‌ها همچنین یک توپولوژی آزمایشی را پیشنهاد کردند که برای ارزیابی برنامه های \لر{ONOS} استفاده می شود\مرجع{christos2019}.

\زیرقسمت{عملکرد}
این موضوع به جنبه‌های کمّی رفتاری آزمون می‌پردازد. نتایج نشان می‌دهد که مطالعات اولیه در این زمینه‌‌ بر روی آزمون عملکرد، عمدتاً در مرحله‌ی تولید توسعه برنامه کاربردی مبتنی بر معماری میکروسرویس تمرکز دارد. 
کامارگو\زیرنویس{Camargo} و همکاران. رویکردی برای ارزیابی عملکرد میکروسرویس به تنهایی ارائه کردند آن‌ها با یکپارچه‌سازی آزمون با میکروسرویس روش خود را توضیح داده‌اند\مرجع{camargo2016}.

علاوه بر این، شارما\زیرنویس{Sharma} و همکاران. با استفاده از مدل تحلیلی و اجرای آزمایش‌های بستر آزمون، عملکرد بین برنامه‌های یکپارچه و میکروسرویسی را در زمینه مجازی‌سازی عملکرد شبکه مورد آزمون و مقایسه قرار دادند\مرجع{sharma2019}.

\زیرقسمت{آزمون مبتنی بر مدل}
 این زمینه پژوهشی، آن دسته از مطالعات اولیه را که در مورد رویکردهای آزمون مبتنی بر مدل برای برنامه‌های کاربردی مبتنی بر معماری میکروسرویس بحث می‌کنند، جمع آوری می‌کند. 
 
 به عنوان مثال، کامیلی\زیرنویس{Camili} و همکاران. یک چارچوب رسمی مبتنی بر مدل‌های شبکه پتری ارائه می‌کند که در زمینه تست میکروسرویس قابل اجرا است. در این پژوهش یک آنتولوژی\پاورقی{Ontology} رسمی مبتنی بر شبکه‌های پتری برای جریان‌های فرآروند مبتنی بر ریزسرویس‌های مشخص شده با استفاده از زبان هماهنگ‌سازی کنداکتور ارائه شده است. کنداکتور یک زبان خاص دامنه مبتنی بر JSON است که توسط نتفلیکس طراحی شده است. همچنین یک معناشناسی صوری از ترجمه‌ی توصیف‌ها در کنداکتور به مدل‌های شبکه پتری که بر پایه‌ی زمان هستند ارائه می‌شود. این مدل‌ها یا شبکه‌های پتری، از تعریف محدودیت‌های زمانی پشتیبانی می‌کنند. این نوع از مدل‌های مبتنی بر شبکه‌ی پتری را می توان برای اهداف صحت‌سنجی به کمک کامپیوتر با استفاده از تکنیک‌های شناخته‌شده‌ی پیاده‎سازی شده توسط ابزارهای قدرتمند بررسی مدل آماده به کار، استفاده کرد\مرجع{camilli2018}. 
 
شولز\زیرنویس{Schulz} و همکاران یک رویکرد آزمون مبتنی بر مدل را برای تولید مدل‌های فشار کاری برای آزمایش فشار یک یا چند میکرسرویس مشخص معرفی کرده‌اند\مرجع{schulz2019}.


\شروع{لوح}[t]
\وسط‌چین
\شرح{نمونه‌ای از جهش‌یافته‌های یک برنامه}



\شروع{جدول}{|p{0.2\linewidth} | p{0.7\linewidth} | p{0.05\linewidth}|}
\خط‌پر
  \toprule
  زمینه‌ی پژوهش & نکات کلیدی پژوهش & پژوهش   \\
  \midrule
  آزمون خودکار  & یک رویکرد مبتنی بر مشخصات رسمی برای استخراج موارد آزمایشی (به عنوان مثال، موارد آزمون پذیرش) برای تست میکروسرویس خودکار & \مرجع{quenum2018}  \\
            & طرح مبتنی بر گراف وابستگی سرویس‌ها برای تحلیل، آزمون و استفاده مجدد از میکروسرویس‌ها   & \مرجع{ma2018} \\
\خط‌پر
  معماری  & آزمون انعطاف‎پذیری میکروسرویس‎ها در زیرساخت های تولید & \مرجع{heorhiadi2016}  \\
            & آزمون عملکردهای سیستم کمک راننده پیشرفته مبتنی بر معماری میکروسرویس (\لر{ADAS})    & \مرجع{lotz2019}  \\
\خط‌پر
  توسعه-عملیات و یکپارچه‌سازی مستمر  & آزمون رگرسیون برنامه‌های کاربردی مبتنی بر معماری میکروسرویس در تحویل مداوم   & \مرجع{kargar2018}  \\
            & آزمون مداوم سیستم عامل شبکه باز ONOS   & \مرجع{christos2019}  \\
\خط‌پر
  عملکرد  & ارزیابی عملکردی که هر میکروسرویس می تواند ارائه دهد   & \مرجع{camargo2016}  \\
            & آزمایش و مقایسه عملکرد برنامه‌های یکپارچه و میکروسرویس در NFV   & \مرجع{sharma2019}  \\
\خط‌پر
  مبتنی بر مدل  & شبکه‌های پتری به عنوان پایه‌ی آزمون‌های مبتنی بر مدل میکروسرویس‌ها   & \مرجع{camilli2018}  \\
            & تولید مدل‌های فشار کاری مبتنی بر سشن برای آزمون فشار میکروسرویس‌ها   & \مرجع{schulz2019}  \\
\خط‌پر
\پایان{جدول}

\برچسب{جدول:تقریب‌پذیری}
\پایان{لوح}







