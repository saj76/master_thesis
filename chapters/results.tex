
\فصل{روش پیشنهادی}

\قسمت {چارچوب کلی}
چارچوب کلی روش پیشنهادی ما بدین صورت است که در ابتدا گردش‌های کاری میان میکروسرویس‌های تشکیل‌دهنده‌ی نرم‎افزار را با استفاده از زبان یاول مدل می‎کنیم و سپس این مدل‌ها را بر اساس روش ترجمه‌ای اثبات شده، به شکل توصیف صوری در زبان الوی ترجمه می‌کنیم. بعد از آن بر روی توصیفات صوری از میکروسرویس‌ها و ارتباطات آن‌ها تحلیل‌هایی انجام می‌دهیم تا از وجود موقعیت‌های ناخواسته و غیرمجاز در نرم‌افزار در زمان اجرای نرم‌افزار جلوگیری کنیم. در نهایت از این توصیفات صوری نرم‎افزار استفاده می‌کنیم و با معیار RACC موارد آزمون را برای آن تولید می‌کنیم و در نهایت موارد آزمون به دست آمده را با روشی اجرا می‌کنیم. 
تمام این فرآیند، پس از مدل‌سازی میکروسرویس‌ها در زبان یاول، توسط طراحان نرم‎افزار، تا ایجاد موارد آزمون و اعمال آن‌ها بر روی مدل صوری به صورت خودکار در ابزار مدل‌سازی یاول انجام می‌شود.

در ادامه به تشریح گام‌های ذکر شده در روش پیشنهادی می‌پردازیم. برای روشن‌تر شدن روش پیشنهادی، نمونه‌ای واقعی از یک سیستم نرم‌افزاری با معماری میکروسرویس را بیان می‌کنیم و در هر گام، روش تشریحی خود را بر روی این نمونه‌ی واقعی اعمال می‌کنیم.
نمونه‌ای که روش پیشنهادی را به مرور بر روی آن اعمال می‌کنیم، خدمت پرداخت با کارت اعتباری است که یک خدمت درشت‌دانه محسوب می‌شود و می‌توان گفت که این خدمت با همکاری چند میکروسرویس ریزدانه‌تر محقق می‌شود؛ در این خدمت با آغاز فراروند پرداخت، دو میکروسرویس به صورت موازی به بررسی مورد خرید و کارت اعتباری خریدار می‌پردازند، اگر هر دوی این میکروسرویس‌ها مورد بررسی خود را تایید کنند، میکروسرویسی دیگر به میزان ارزش مورد خرید، از اعتبار کارت اعتباری خریدار کم می‌کند و فراروند پرداخت به اتمام می‌رسد؛ اما در صورتی که کارت اعتباری یا مورد خرید،  تایید نشوند بلافاصله فراروند پرداخت تمام می‌شود. بدون این که از کارت اعتباری مبلغی کم شود. شکل~\رجوع{شکل:گردش کاری پرداخت} نشان‌دهنده‌ی گردش کاری این خدمت است.
\شروع{شکل}[t]
\centerimg{sample-payment-workflow}{14cm}
\vspace{0.5em}
\شرح{گردش کاری خدمت پرداخت}
\برچسب{شکل:گردش کاری پرداخت}
\پایان{شکل}

\قسمت{مدل‌سازی سیستم‌های پیچیده با معماری میکروسرویس با یاول}
امروزه خیلی از برنامه‎ها شامل مجموعه‌ای از سرویس‌ها هستند هر میکروسرویس به طور مستقل توسعه یافته، مستقر و مدیریت می شود. 
همکاری میکروسرویس‌ها با یکدیگر هدف برنامه را محقق می‌کند؛ هماهنگی و تعامل میان میکروسرویس‌ها با به کارگیری یکی از دو رویکرد Orchestration یا  choreography انجام می‌شود.
به دلیل ذات غیرمتمرکز میکروسرویس‌ها به نظر می‌رسد استفاده از رویکرد Choreography برای ترکیب\پاورقی{Compostion} آن‌ها مناسب تر باشد. 
در Choreography، هر میکروسرویس به طور مستقل کار می کند، در حالی که، در ارکستراسیون، یک کنترل‌کننده\پاورقی{Controller} وجود دارد که تعاملات سرویس را هماهنگ می کند.\مرجع{valderas2020}\مرجع{nadeem2022} 

در این پایان‌نامه، ما به برنامه‌هایی که از رویکرد Choreography برای ساختن میکروسرویس‌ها استفاده می‌کنند، می‌پردازیم و روش خود را برای آن‌ها پیشنهاد می‌کنیم. 
استفاده از زبان‌ مدل‌سازی یاول که یک زبان مدیریت فراروند کسب و کاری است یکی از رویکردها برای مدل‌سازی از این تعاملات بین میکروسرویس‎ها در یک برنامه‌ی پیچیده است.\مرجع{valderas2020}
در روش پیشنهادی ما ابتدا یک برنامه با استفاده از یاول مدل می‌شود، میکروسرویس‌ها واحد‌های مستقلی هستند که هر کدام جزئی از کل کار برنامه را بر عهده دارند و گردش کار بین میکروسرویس‌ها معمولا با رد و بدل شدن پیام بین آن‌ها انجام می‌پذیرد \مرجع{heorhiadi2016}،
 در Choreography این ارتباط بدون واسطه و مستقیم انجام می‌شود. 
 
 
 در یاول کوچک‌ترین واحدهای کاری مستقل، وظایف هستند و می‌توان آن‌ها را معادل میکروسرویس‌ها در یک برنامه در نظر گرفت. 
 همچنین در یاول ارتباط میان وظایف با جریان‌ها برقرار می‌شوند، \مرجع{yawlbook} می‌توان برای نمایش ارسال پیام‌ها میان میکروسرویس‌ها از جریان‌ها استفاده کرد. 
 از انشعاب‌ها و اتصال‌ها برای هدایت گردش کار در یاول استفاده می‌شود، در آن‌ها با توجه به متغیر‌های ورودی و خروجی و شروط اعمال شده بر روی آن‌ها گردش کار توسط وظایف هدایت می‌شود. 
 مقادیر متغیرها در برنامه را خروجی میکروسرویس‌ها تعیین می‌کنند و جریان کنترل در برنامه با توجه به خروجی هر میکروسرویس بین آن‌ها گردش می‌کند. شکل ~\رجوع{شکل:پرداخت یاول} مدل‌سازی از برنامه‌ی پرداخت، به زبان یاول است.
\شروع{شکل}[t]
\centerimg{payment-yawl}{14cm}
\vspace{0.5em}
\شرح{مدل خدمت پرداخت در یاول}
\برچسب{شکل:پرداخت یاول}
\پایان{شکل}

در این مثال وظیفه‌ی "آغاز پرداخت" انشعاب از نوع "و" دارد زیرا بعد از اتمام این وظیفه بلافاصله هر دو وظیفه‌ی "بررسی کارت اعتباری" و "بررسی موارد خرید" شروع به کار می‌کنند؛ 
خروجی این دو میکروسرویس تعیین‌کننده‌ی جریان کنترل در برنامه هستند، اگر هر دو میکروسرویس‌ موارد بررسی خود را تایید کنند کنترل به وظیفه‌ی "شارژ کردن کارت اعتباری" می‌رسد؛
 اما اگر هر دو یا یکی از دو میکروسرویس بررسی‌کننده، خروجی تایید نداشته باشند کنترل به وظیفه‌ی "پایان پرداخت" می‌رسد، در نتیجه انشعاب در دو وظیفه‌ی بررسی‌کننده، از نوع "یای انحصاری\پاورقی{Xor}" است و اتصال در وظیفه‌ی "شارژ کردن کارت اعتباری" از نوع "و" است همچنین وظیفه‌ی "پایان پرداخت" اتصال از نوع "یا" دارد زیرا که یک یا چند جریان ورودی آن می‌تواند آن را فعال کنند تا شروع به کار کند.
  در نهایت نیز پس از اتمام وظیفه‌ی "پایان پرداخت" کار برنامه نیز تمام می‌شود. 
  
  \قسمت{خودکارسازی ترجمه‌ی مدل‌های یاول به الوی}
  
  \زیرقسمت{توصیف مدل‌ها در زبان الوی}
  
در این پایان‌نامه برای ترجمه‌ی مدل‌های یاول به توصیفات صوری در زبان الوی، از روشی که در پژوهش ریواده و همکاران انجام شده است استفاده کردیم؛
 در روشی که ریواده و همکاران برای ترجمه ارائه کرده‌اند، ساختار مدل‌ها در یاول به دو بخش ایستا و پویا تقسیم‌بندی شده‌است و برای هر یک از موجودیت‌ها در هر دسته معادلی در الوی ذکر شده‌است. 
 همچنین برای ویژگی‌های ذاتی مدل‌های گردش کاری در یاول مانند هم‌بند بودن گراف، در الوی حقیقت\پاورقی{fact}‌هایی تعریف شده‌است.
  در نهایت روشی برای ترجمه‌ی مدل‌ها به دست آمده و سپس با نه قضیه نشان داده‌است که ترجمه‌ی به دست آمده از مدل‌های گردش کاری در یاول به زبان الوی کامل و صحیح\پاورقی{sound} هستند.
  
بخش ایستا در مدل‌ها، همان مفاهیم و مولفه‌های موجود در زبان یاول هستند. در پژوهش ریواده این بخش شامل وظیفه\پاورقی{task}، شرط ورودی\پاورقی{input condition}، شرط خروجی\پاورقی{output condition} و شرط است؛ 
برای هر کدام از این مولفه‌ها در الوی یک معادل در قالب نشان\پاورقی{signature} آورده شده است. همچنین رفتار انواع پیوند\پاورقی{join}ها و انشعاب\پاورقی{split}‌ها نیز در قالب حقیقت‌ها بیان شده اند. 
علاوه بر این‌ها بخش ایستا در ترجمه‌ی تولید‌شده شامل تعریف حالت\پاورقی{state} در یک گردش کار نیز می‌شود به تعریف حالت در الوی ترتیب اضافه شده‌است، 
این کار اجازه می‌دهد بتوان تغییر حالت‌ها در زمان را مدل کرد. ترتیب حالت‌ها توسط ماژول کتابخانه util/ordering ارائه می شود. این ماژول عمومی است - یعنی می‌تواند به مجموعه‌ای از هر نوع ترتیب بدهد - بنابراین وقتی باز می‌شود باید با یک نوع (در این مورد، حالت) نمونه‌سازی\پاورقی{instantiate} شود. (منبع کتاب الوی)
مجموعه‌ی توکن مجموعه‌ای از وظایف یا شرط ورودی یا شرط خروجی است که کاری در آن‌ها در حال انجام است در هر حالت از گردش کار، توکن در یک یا چند وظیفه وجود دارد و با تغییر حالت، در میان وظیفه‌ها جابجا می‌شود. در واقع و تغییر حالت متناظر با تغییر مجموعه‌ی توکن است. 
بخش پویا مرتبط با معماری میکروسرویسی است که در یاول مدل شده است. 

برای ترجمه‌ی مدل تعریف شده به الوی، برای هر وظیفه که معادل یک میکروسرویس است، حقیقت یا حقیقت‌هایی در الوی تعریف می‌کنیم و در آن(ها) ویژگی‌های وظیفه شامل نام، نوع پیوند، نوع انشعاب و جریان‌های خروجی آن را ذکر می‌کنیم؛ 
همچنین مسندهایی که در جریان‌های خروجی وظیفه تعریف شده‌اند را نیز بسته به نوع انشعاب، در گزاره‌های شرطی ذکر می‌کنیم. 
بخش‌ پویا در واقع نشان‌دهنده‌ی میکروسرویس‌ها و نحوه‌ی ارتباط آن‌ها با یکدیگر هستند و شامل وظیفه و جریان‌های بین آن‌ها می‌شود. 
ما در این پایان‌نامه موارد جدیدی به ترجمه اضافه کردیم که در ادامه به آن‌ها می‌پردازیم؛ در ترجمه‌ای که از مدل‌های یاول تولید می‌کنیم، شامل توصیف ناحیه‌ی لغو در بخش ایستا نیز می‌باشد، 
همچنین در تعریف وظیفه‌ها نیز مجموعه‌ی وظیفه‌هایی که در ناحیه‌ی لغو آن وظیفه وجود دارند تعریف می‌شود. 
متغیرهای موجود در مدل گردش کاری که شامل ورودی خروجی وظیفه‌ها و معادل ورودی و خروجی میکروسرویس‌ها هستند، در تعریف حالت ذکر می‌شوند. 
مقادیر متغیرها در هر حالت امکان تغییر دارند و رفتار میکروسرویس‌ها در هر حالت بسته به مقدار متغیرها و شرایط درونی میکروسرویس می‌تواند تغییر کند.
قطعه‌ی کدی که در پیوست~\رجوع{شکل:پرداخت یاول} ترجمه‌ی نمونه‌‌ای است که در شکل~\رجوع{شکل:پرداخت یاول} آمده است: 

  
  
