
\فصل{روش پیشنهادی}
\label{فصل:روش پیشنهادی}

\قسمت {چارچوب کلی}
چارچوب کلی روش پیشنهادی ما بدین صورت است که در ابتدا گردش‌های کاری میان میکروسرویس‌های تشکیل‌دهنده‌ی نرم‎افزار را با استفاده از زبان یاول مدل می‎کنیم. سپس این مدل‌ها را بر اساس روش ترجمه‌ای اثبات شده، به یک توصیف صوری در زبان الوی ترجمه می‌کنیم. بعد از آن بر روی توصیف صوری از میکروسرویس‌ها و ارتباطات آن‌ها تحلیل‌هایی صوری به زبان الوی انجام می‌دهیم تا وجود موقعیت‌های ناخواسته و غیرمجاز در اجرای گردش کاری نرم‌افزار را بررسی کنیم. به علاوه با استفاده از توصیفات صوری مذکور و بر اساس معیار پوشش گزاره‌ی فعال محدود، نیازمندی‌های آزمون و سپس برای برآورده کردن هر یک از نیازمندی‌های آزمون موارد آزمون را تولید می‌کنیم و در انتها موارد آزمون به دست آمده که آن‌ها نیز توصیفاتی صوری هستند را به توصیف اصلی اضافه می‌کنیم؛ سعی می‌کنیم یک نمونه از توصیف به‌دست آمده بسازیم، در صورت سازگار ماندن توصیف، این نمونه به دست می‌آید و موفقیت‌آمیز بودن اجرای آزمون را نشان می‌دهد. 

روند کلی روش ارائه‌شده در شکل آورده شده است.
شکل~\رجوع{شکل: گردش کاری یک برنامه‌ی نمونه برای خرید برخط} نشان‌دهنده‌ی گردش کاری این برنامه است.
\شروع{شکل}[t]
\centerimg{method-flow}{12cm}
\vspace{0.3em}
\شرح{روند کلی روش ارائه‌شده}
\برچسب{شکل: روند کلی روش ارائه‌شده}
\پایان{شکل}

تمام این روند، پس از مدل‌سازی میکروسرویس‌ها در زبان یاول توسط طراحان نرم‎افزار، تا ایجاد موارد آزمون و اعمال آن‌ها بر روی مدل صوری به صورت خودکار در ابزار مدل‌سازی یاول انجام می‌شود.

در ادامه به تشریح گام‌های ذکر شده در روش پیشنهادی می‌پردازیم. اما برای روشن‌تر شدن این روش یک نمونه‌‌ی برگرفته‌شده از یک سیستم نرم‌افزاری با معماری میکروسرویس در دنیای واقع را می‌آوریم و در هر گام، روش تشریحی خود را بر روی این نمونه‌ی واقعی اِعمال می‌کنیم. 

نمونه‌ای که گفته ‌خواهد شد، یک برنامه‌ی خرید به صورت برخط است که از میکروسرویس‌های مستقل برای ارائه‌ی خدمت خود تشکیل شده‌ است. این میکروسرویس‌ها شامل میکروسرویس ورود با استفاده از گوگل\زیرنویس{Google}، ورود با استفاده از توئیتر\زیرنویس{Twitter} ورود با استفاده از لینکدین\زیرنویس{Linkedin}، سرویس خرید (انتخاب موارد خرید)، پرداخت با کارت اعتباری، پرداخت با پی‌پل\زیرنویس{PayPal} و تایید پرداخت است. می‌توان گفت که این خدمت با همکاری چند میکروسرویس ریزدانه‌تر محقق می‌شود؛ در این برنامه با آغاز روند خرید و اولین درخواست کاربر، اگر برنامه تشخیص بدهد که  نیاز به بررسی امنیتی درخواست کاربر وجود دارد میکروسرویس بررسی‌کننده‌ی امنیت این وظیفه را انجام می‌دهد و اگر درخواست را غیر مجاز یا خرابکارانه تشخیص دهد، بلافاصله فرآیند برنامه به اتمام می‌رسد و کاربر مجبور است درخواست دیگری برای آغاز خرید ارسال کند؛ اما در صورتی که درخواست از جنبه‌ی امنیتی تایید شود، کاربر باید احراز هویت خود را انجام دهد و این هدف توسط یکی از سه میکروسرویس ورود با استفاده از گوگل، ورود با استفاده از توئیتر ورود با استفاده از لینکدین به انتخاب کاربر، محقق می‌شود. اگر برنامه، بررسی امنیتی درخواست کاربر را لازم نداند، احراز هویت کاربر بلافاصله پس از درخواست کاربر صورت می‌پذیرد و اگر احراز هویت کاربر به درستی صورت نگیرد کاربر به حالت ابتدایی خود پس از درخواست شروع برنامه منتقل می‌شود و روند گفته‌‌شده در تلاش بعدی کاربر تکرار می‌شود؛ در غیر این صورت پس از احراز هویت کاربر، انتخاب کالاها و پر کردن سبد خرید توسط میکروسرویس نمایش و انتخاب کالاها، در تعامل با کاربر انجام می‌شود. 

پس از انتخاب کالاها، کاربر باید برای تکمیل خرید خود سبد خود را تسویه حساب کند، این وظیفه نیز توسط یکی از دو میکروسرویس پرداخت یعنی پرداخت با کارت اعتباری یا پرداخت با پی‌پل انجام می‌شود. اگر پرداخت موفقیت‌آمیز باشد جریان کنترل برنامه به میکروسرویس انتخاب کالا باز گردانده ‌ می‌شود و اگر پرداخت موفق باشد نیز میکروسرویس اتمام خرید که نحوه‌ی دریافت و تاییدیه‌ی پرداخت کاربر را به او نمایش می‌دهد خدمت خرید را به اتمام می‌رساند.

شکل~\رجوع{شکل: گردش کاری یک برنامه‌ی نمونه برای خرید برخط} نشان‌دهنده‌ی گردش کاری این برنامه است.
\شروع{شکل}[t]
\centerimg{shopping-example-workflow}{14cm}
\vspace{0.5em}
\شرح{گردش کاری یک برنامه‌ی نمونه برای خرید برخط}
\برچسب{شکل: گردش کاری یک برنامه‌ی نمونه برای خرید برخط}
\پایان{شکل}



\قسمت{مدل‌سازی سیستم‌های پیچیده با معماری میکروسرویس با یاول}
امروزه خیلی از برنامه‎ها شامل مجموعه‌ای از سرویس‌ها هستند هر میکروسرویس به طور مستقل توسعه یافته، مستقر و مدیریت می شود. 
همکاری میکروسرویس‌ها با یکدیگر هدف برنامه را محقق می‌کند؛ هماهنگی و تعامل میان میکروسرویس‌ها با به کارگیری یکی از دو رویکرد ارکستراسیون یا  موزون انجام می‌شود.
به دلیل ذات غیرمتمرکز میکروسرویس‌ها به نظر می‌رسد استفاده از رویکرد موزون برای ترکیب\پاورقی{Compostion} آن‌ها مناسب تر باشد. 
در سبک موزون، هر میکروسرویس به طور مستقل کار می کند، در حالی که، در ارکستراسیون، یک کنترل‌کننده\پاورقی{Controller} وجود دارد که تعاملات سرویس را هماهنگ می کند.\مرجع{valderas2020}\مرجع{nadeem2022} 

پایان‌نامه، ما روشی برای آزمون هر دو این سبک‌ها ارائه می‌کنیم اما نمونه‌ای که برای روشن‌تر شدن موضوع استفاده شده‌است از رویکرد Orchestration برای ساختن معماری میکروسرویسی خود استفاده می‌کند. 

استفاده از زبان‌ مدل‌سازی یاول که یک زبان مدیریت فراروند کسب و کاری است یکی از رویکردها برای مدل‌سازی از این تعاملات بین میکروسرویس‎ها در یک برنامه‌ی پیچیده است.\مرجع{valderas2020}

در روش پیشنهادی ما ابتدا یک برنامه با استفاده از یاول مدل می‌شود، میکروسرویس‌ها واحد‌های مستقلی هستند که هر کدام جزئی از کل کار برنامه را بر عهده دارند و گردش کار بین میکروسرویس‌ها معمولا با رد و بدل شدن پیام بین آن‌ها انجام می‌پذیرد \مرجع{heorhiadi2016}،
 در سبک موزون این ارتباط بدون واسطه و مستقیم انجام می‌شود. 
 
 در یاول کوچک‌ترین واحدهای کاری مستقل، وظایف هستند و می‌توان آن‌ها را معادل میکروسرویس‌ها در یک برنامه در نظر گرفت. 
 همچنین در یاول ارتباط میان وظایف با جریان‌ها برقرار می‌شوند، \مرجع{yawlbook} می‌توان برای نمایش ارسال پیام‌ها میان میکروسرویس‌ها از جریان‌ها استفاده کرد. 
 از انشعاب‌ها و اتصال‌ها برای هدایت گردش کار در یاول استفاده می‌شود، در آن‌ها با توجه به متغیر‌های ورودی و خروجی و شروط اعمال شده بر روی آن‌ها گردش کار توسط وظایف هدایت می‌شود. 
 مقادیر متغیرها در برنامه را خروجی میکروسرویس‌ها تعیین می‌کنند و جریان کنترل در برنامه با توجه به خروجی هر میکروسرویس بین آن‌ها گردش می‌کند. شکل ~\رجوع{شکل:مدل گردش کاری برنامه‌ی خرید برخط در یاول} مدل‌سازی از برنامه‌ی پرداخت، به زبان یاول است.
\شروع{شکل}[t]
\centerimg{shopping-example-yawl}{14cm}
\vspace{0.5em}
\شرح{مدل گردش کاری برنامه‌ی خرید برخط در یاول}
\برچسب{شکل:مدل گردش کاری برنامه‌ی خرید برخط در یاول}
\پایان{شکل}

در این مثال وظیفه‌ی "بررسی‌کننده‌ی امنیتی" انشعاب از نوع "یا" دارد و بعد از اتمام این وظیفه با توجه به پاسخ آن یا "شروع فرآیند ورود" بلافاصله آغاز می‌شود و یا این که بررسی‌کننده‌ی امنیتی خدمت برنامه، را منع می‌کند. همچنین وظیفه‌ی "شروع فرآیند ورود" دارای انشعاب از نوع "یای انحصاری" است زیرا کاربر تنها با یک روش از روش‌های سه‌گانه‌ی ورود ممکن (یعنی "ورود با گوگل"، "ورود با لینکدین" و "ورود با توئیتر")، می‌تواند احراز هویتش را انجام دهد و وارد حساب کاربری خود شود. اگر که پاسخ میکروسرویس‌هایی که عمل احراز هویت را انجام می‌دهند تایید هویت کاربر باشد، ورود کاربر موفقیت آمیز است و جریان کنترل برنامه به میکروسرویس "خرید (انتخاب موارد)" می‌رسد و در غیر این صورت جریان کنترل به ابتدای برنامه، یعنی "شروع خرید" باز می‌گردد.

پس از انتخاب موارد خرید توسط میکروسرویس "خرید"، بسته به انتخاب کاربر، یکی از میکروسرویس‌های پرداخت ("پرداخت با کارت اعتباری" و "پرداخت با پی‌پل") شروع به کار می‌کنند در نتیجه انشعاب در وظیفه‌ی "شروع پرداخت" از نوع "یای انحصاری" است. اگر پاسخ میکروسرویس‌ پرداخت، پرداخت موفقیت آمیز باشد، روند خرید تکمیل می‌شود و خدمت برنامه به پایان می‌رسد؛ اما اگر "پرداخت" ناموفق باشد، جریان کنترل به وظیفه‌ی "شروع پرداخت" بازگردانده می‌شود که در حین آن وظیفه کاربر مجددا به انتخاب سرویس پرداخت می‌پردازد. با توجه به عملکرد مورد انتظار از برنامه و شرایط گفته‌شده، وظیفه‌ی "شروع پرداخت" پیوند از نوع "یای انحصاری" دارد. 

  \قسمت{خودکارسازی ترجمه‌ی مدل‌های یاول به الوی}
  
  \زیرقسمت{توصیف مدل‌ها در زبان الوی}
  
در این پایان‌نامه برای ترجمه‌ی مدل‌های یاول به توصیفات صوری در زبان الوی، از روشی که در پژوهش ریواده و همکاران انجام شده است\مرجع{rivadeh2022} استفاده کردیم؛
 در روشی که ریواده و همکاران برای ترجمه ارائه کرده‌اند، ساختار مدل‌ها در یاول به دو بخش ایستا و پویا تقسیم‌بندی شده‌است و برای هر یک از موجودیت\پاورقی{‌Entity}‌ها در هر دسته معادلی در الوی ذکر شده‌است. 
 همچنین برای ویژگی‌های ذاتی مدل‌های گردش کاری در یاول مانند هم‌بند بودن گراف، در الوی حقیقت\پاورقی{fact}‌هایی تعریف شده‌است.
  در نهایت روشی برای ترجمه‌ی مدل‌ها به دست آمده و سپس با نه قضیه نشان داده‌است که ترجمه‌ی به دست آمده از مدل‌های گردش کاری در یاول به زبان الوی کامل و صحیح\پاورقی{sound} هستند.
  
بخش ایستا در مدل‌ها، همان مفاهیم و مولفه‌های موجود در زبان یاول هستند. در پژوهش ریواده این بخش شامل وظیفه\پاورقی{task}، شرط ورودی\پاورقی{input condition}، شرط خروجی\پاورقی{output condition} و شرط است؛ 
برای هر کدام از این مولفه‌ها در الوی یک معادل در قالب نشان\پاورقی{signature} آورده شده است. همچنین رفتار انواع پیوند\پاورقی{join}ها و انشعاب\پاورقی{split}‌ها نیز در قالب حقیقت‌ها بیان شده اند. 
علاوه بر این‌ها بخش ایستا در ترجمه‌ی تولید‌شده شامل تعریف حالت\پاورقی{state} در یک گردش کار نیز می‌شود به تعریف حالت در الوی ترتیب اضافه شده‌است، 
این کار اجازه می‌دهد بتوان تغییر حالت‌ها در زمان را مدل کرد. ترتیب حالت‌ها توسط ماژول کتابخانه util/ordering ارائه می شود. این ماژول عمومی است - یعنی می‌تواند به مجموعه‌ای از هر نوع ترتیب بدهد - بنابراین وقتی باز می‌شود باید با یک نوع (در این مورد، حالت) نمونه‌سازی\پاورقی{instantiate} شود\مرجع{alloybook}.

مجموعه‌ی توکن مجموعه‌ای از وظایف یا شرط ورودی یا شرط خروجی است که کاری در آن‌ها در حال انجام است در هر حالت از گردش کار، توکن در یک یا چند وظیفه وجود دارد و با تغییر حالت، در میان وظیفه‌ها جابجا می‌شود. در واقع و تغییر حالت متناظر با تغییر مجموعه‌ی توکن است. 
بخش پویا مرتبط با معماری میکروسرویسی است که در یاول مدل شده است. 

برای ترجمه‌ی مدل تعریف شده به الوی، برای هر وظیفه که معادل یک میکروسرویس است، حقیقت یا حقیقت‌هایی در الوی تعریف می‌کنیم و در آن(ها) ویژگی‌های وظیفه شامل نام، نوع پیوند، نوع انشعاب و جریان‌های خروجی آن را ذکر می‌کنیم؛ 
همچنین مسندهایی که در جریان‌های خروجی وظیفه تعریف شده‌اند را نیز بسته به نوع انشعاب، در گزاره‌های شرطی ذکر می‌کنیم. 
بخش‌ پویا در واقع نشان‌دهنده‌ی میکروسرویس‌ها و نحوه‌ی ارتباط آن‌ها با یکدیگر هستند و شامل وظیفه و جریان‌های بین آن‌ها می‌شود.
 
ما در این پایان‌نامه موارد جدیدی به ترجمه اضافه کردیم که در ادامه به آن‌ها می‌پردازیم؛ در ترجمه‌ای که از مدل‌های یاول تولید می‌کنیم، شامل توصیف ناحیه‌ی لغو در بخش ایستا نیز می‌باشد، 
همچنین در تعریف وظیفه‌ها نیز مجموعه‌ی وظیفه‌هایی که در ناحیه‌ی لغو آن وظیفه وجود دارند تعریف می‌شود. 
متغیرهای موجود در مدل گردش کاری که شامل ورودی خروجی وظیفه‌ها و معادل ورودی و خروجی میکروسرویس‌ها هستند، در تعریف حالت ذکر می‌شوند. 
مقادیر متغیرها در هر حالت امکان تغییر دارند و رفتار میکروسرویس‌ها در هر حالت بسته به مقدار متغیرها و شرایط درونی میکروسرویس می‌تواند تغییر کند.
قطعه‌ی کدی که در پیوست ۱ آمده، ترجمه‌ی نمونه‌‌ای است که در شکل~\رجوع{شکل:مدل گردش کاری برنامه‌ی خرید برخط در یاول} آمده است.


\زیرقسمت{جای‌گذاری روش در ابزار ویرایش یاول}
روش پیشنهادی گفته شده را در ابزار ویرایشگر یاول جاسازی کردیم و کار ترجمه در این ابزار به صورت خودکار انجام می‌شود و در خروجی به طراح نرم‌افزار نمایش داده می‌شود. برای تولید ترجمه به این صورت عمل می‌کنیم: قسمت‌های ایستا به صورت ثابت و ایستا به ترجمه اضافه می‌شوند اما برای ترجمه‌ی خودکار مولفه‌های متغیر در مدل‌ها ما پیش‌‎پردازشی بر روی فرم استاندارد ذخیره‌شده‌ی مدل‌ یاول انجام می‌دهیم و سپس با روشی هر کدام از مولفه‌های پویا را تجزیه و تحلیل می‌کنیم و آن‌ها را به اشیائی از کلاس‌های تعریف شده در زبان جاوا تبدیل می‌کنیم و سپس در انتهای کار ترجمه از هر کدام از آن اشیاء، توصیف آن‌ها را به زبان الوی پرس و جو می‌کنیم و سپس آن‌ها را تجمیع و به قسمت‌های ایستای توصیف اضافه می‎کنیم. در نهایت، توصیف به دست آمده شامل تعاریف اجزای ایستای مدل‌ها که شامل انواع پیوندها و انشعاب‌ها و همچنین تعاریف وظیفه و شرط ورودی و شرط خروجی است.

\قسمت{تحلیل صوری مدل‌ها}
در این بخش به تحلیل‌هایی که بر روی مدل گردش کاری در میکروسرویس‌ها انجام می‌دهیم، می‌پردازیم. 

تحلیل‌های که در ادامه می‌آیند، با بررسی درستی اظهار در الوی انجام می‌شود به این صورت که یک اظهار به کل تعریف گردش کاری افزوده می‌شود و سپس توصیف حاصل‌شده را به موتور الوی می‌دهیم. اظهارها به صورت کلی به این فرمت نوشته می‌شوند که "در هیچ نمونه‌ای از توصیف موجود حالت غیرمجاز وجود ندارد" اگر موتور الوی بتواند مثال نقضی برای این اظهار پیدا کند، پیدا شدن نمونه‌ای که این اظهار را نقض کند، در کاربرد ما به منزله‌ی وجود حالتی غیرمجاز است که ساختار گردش کاری باعث پیدایش آن شده است.

\شروع{فقرات}
\فقره بررسی پیوند از نوع "یا" در حلقه: 

بنا بر تعریف، این وضعیت که یک وظیفه که دارای پیوند از نوع "یا" است در حلقه قرار گیرد، غیرمجاز است؛ در این حالت، وظیفه‌ای که دارای پیوند از نوع "یا" است منتظر مشخص شدن وضعیت شاخه‌ی ورودی خود است در حالی که مشخص شدن وضعیت شاخه‌ی ورودی به خروجی همین وظیفه بستگی دارد و در نتیجه گردش کار دچار نوعی بن‌بست می‌شود. ما برای تشخیص این حالت غیرمجاز در روش پیشنهادی خود، تحلیلی را ارائه کردیم که با بررسی ساختار گردش کاری ورودی که آیا وظیفه‌ای باعث ایجاد این حالت غیرمجاز می‌شود یا خیر.

همان‌طور که در بالاتر شرح داده‌شد، این تحلیل با افزودن یک اظهار و سپس بررسی آن در توصیف انجام‌ می‌شود. اظهار به صورت زیر توصیف می‌شود. 



\شروع{شکل}[H]
\raggedright
\includegraphics[width=9cm]{no-or-join-in-loop}
\vspace{0.5em}
\برچسب{شکل: اظهار تولیدشده برای اطمینان از عدم وجود پیوند از نوع ``یا'' در حلقه}
\پایان{شکل}

مثالی خری برخط را در نظر بگیرید، سناریویی را در نظر بگیرید که طراح برای شروع کار میکروسرویس "شروع خرید"، شرط انجام کار در میکروسرویس "احراز هویت و ورود" قبلی را لحاظ می‌کند و گردش کار مشخص شده در شکل ~\رجوع{شکل:وجود پیوند از نوع ``یا'' در مدل برنامه‌ی خرید برخط در یاول} را تعریف می‌کند. 
\شروع{شکل}[t]
\centerimg{no-or-join-in-loop-yawl}{14cm}
\vspace{0.5em}
\شرح{وجود پیوند از نوع ``یا'' در مدل}
\برچسب{شکل:وجود پیوند از نوع ``یا'' در مدل برنامه‌ی خرید برخط در یاول}
\پایان{شکل}
در این گردش کاری وظیفه‌ی "شروع خرید" که دارای پیوند از نوع "یا" است در انتظار مشخص شدن وضعیت جریان‌های ورودی‌ خود است تا پس از آن کار خود را آغاز کند، در حالی که مشخص شدن وضعیت ورودی‌های آن به فعال بودن با نبودن خروجی وظیفه‌ی "احراز هویت و ورود" بستگی دارد؛ در حالی که مشخص شدن فعال یا غیر فعال بودن خروجی این وظیفه به مشخص بودن وضعیت "آغاز خرید" بستگی دارد؛ در نتیجه گردش کار در وضعیت بن‌بست قرار گرفته است که وضعیتی نامطلوب است و دلیل آن نیز قرار گرفتن وظیفه‌ای با پیوند از وضعیت "یا" در حلقه است.

اظهار تولید شده برای گردش کاری در شکل ~\رجوع{شکل:وجود پیوند از نوع ``یا'' در مدل برنامه‌ی خرید برخط در یاول} ،در زیر آمده است:



\شروع{شکل}[H]
\raggedright
\includegraphics[width=10cm]{or-join-in-loop-shopping}
\vspace{0.5em}
\برچسب{شکل: اظهار تولیدشده برای اطمینان از عدم وجود پیوند از نوع ``یا'' در حلقه در برنامه‌‌ی خرید برخط}
\پایان{شکل}


\فقره قابل دسترس بودن وظایف:

میکروسرویس هایی که در معماری نرم‌افزار استفاده شده‌اند مسئولیت انجام قسمتی از خدمت را به عهده دارند در نتیجه در طول اجرای نرم‌افزار زمانی وجود خواهد داشت که هر میکروسرویس در حال انجام وظیفه‌ای است که به عهده دارد، اگر میکروسرویسی در جریان کنترل نرم‌افزار قابل دسترسی نباشد وجود آن میکروسرویس برای نرم‌افزار غیرضروری است؛ به طوری که گردش کار بدون وجود آن وظیفه‌‌ی غیر قابل دسترس نیز همان رفتاری را نشان می‌دهد که با وجود آن وظیفه دارد.

در این تحلیل ما برای هر وظیفه به دنبال این هستیم که در حداقل یک حالت‌ از گردش کار، وظیفه‌ی مورد بررسی، در مجموعه‌ی توکن آن حالت باشد. این تحلیل نیز با افزودن یک اظهار به توصیف و سپس پیدا کردن مثال نقض برای آن توسط موتور الوی انجام‌ می‌شود. اظهار به صورت زیر نوشته می‌شود. 
\شروع{شکل}[H]
\raggedright
\includegraphics[width=10cm]{reachability-assertion}
\vspace{0.5em}
\برچسب{شکل: اظهار تولیدشده برای اطمینان از قابل دسترس بودن وظیفه}
\پایان{شکل}

اظهار بالا برای تک تک وظایف گردش کاری به صورت متوالی به توصیف گردش کاری افزوده می‌شود و بررسی نیز انجام می‌شود و نتیجه‌ی بررسی آن‌ها به صورت تجمیع شده در خروجی برمی‌گردد.

در مثال خدمت پرداخت، فرض کنید که طراح در میکروسرویس "تاییدکننده‌ی پرداخت" شرط فعال شدن میکروسرویس "اتمام خرید" را این‌گونه تعریف کند:


\شروع{شکل}[H]
\raggedright
\includegraphics[width=10cm]{corrupted-condition-results-inreachablility}
\vspace{0.5em}
\برچسب{شکل: شرط نادرست در جریان خروجی که سبب غیر قابل دسترس بودن وظیفه می‌شود}
\پایان{شکل}

در حالی که مقدار شرط صحیح این‌گونه است:



\شروع{شکل}[H]
\raggedright
\includegraphics[width=10cm]{right-condition-in-output-flow}
\vspace{0.5em}
\برچسب{شکل: شرط درست در جریان خروجی}
\پایان{شکل}



در این صورت هرگز ارزش مسند 
\شروع{شکل}[H]
\raggedright
\includegraphics[width=10cm]{corrupted-predicate}
\vspace{0.5em}
\برچسب{شکل: نمونه مسند اشتباه در جریان خروجی }
\پایان{شکل}
برابر "درست" نخواهد شد. در نتیجه هیچ گاه به وظیفه‌ی "اتمام خرید" در زمان اجرای گردش کار نخواهیم رسید و در حقیقت وظیفه‌ی "اتمام خرید" در گردش کاری طراحی شده غیرقابل دسترسی است. اظهار افزوده شده برای یافتن مثال نقض این گزاره به صورت زیر به توصیف گردش کار افزوده می‌شود. 



\شروع{شکل}[H]
\raggedright
\includegraphics[width=10cm]{reachability-assertion-shopping}
\vspace{0.5em}
\برچسب{شکل: اظهار نوشته‌شده برای قابل دسترس بودن در نمونه‌ی خرید}
\پایان{شکل}
 

\فقره منتظر ماندن دو وظیفه با پیوند از نوع "یا" برای یکدیگر:

در این تحلیل ما پس از بررسی وجود دو وظیفه با پیوند از نوع "یا" در گردش کاری، با تحلیل ساختار بررسی میکنیم که آیا این دو وظیفه در تصمیم گیری برای جلو رفتن گردش کار در خودشان منتظر یکدیگر هستند یا خیر. چرا که با منتظر بودن این دو وظیفه برای مشخص شدن وضعیت دیگری، وضعیت بن‌بست در گردش کاری رخ می‌دهد که غیرمجاز است.

 این تحلیل نیز با افزودن یک اظهار و سپس بررسی آن در توصیف انجام‌ می‌شود. اظهار را به صورت زیر می‌نویسیم. 
 
 
 
\شروع{شکل}[H]
\raggedright
\includegraphics[width=17cm]{two-pending-or-joins-assertion}
\vspace{0.5em}
\برچسب{شکل: اظهار نوشته‌شده برای بررسی منتظر بودن دو پیوند از نوع ``یا''}
\پایان{شکل}


مجددا مثال برنامه‌ی خرید برخط را در نظر بگیرید فرض کنید در این مثال طراح برای نمایش اجرای دو بار "خرید" به صورت پشت سر هم، توسط کاربر گردش کار زیر را طراحی میکند، در این گردش کار، میکروسرویس "آغاز خرید"، دارای پیوند از نوع "یا" است و برای آغاز به کار در انتظار مشخص شدن وضعیت جریان‌های ورودی خود است.در حالی که یکی از ورودی‌های آن میکروسرویس "احراز هویت و ورود" است که خود دارای پیوند از نوع "یا" است و مشخص شدن وضعیت ورودی‌های آن به مشخص شدن وضعیت وظیفه‌ی "آغاز خرید"، بستگی دارد. در نتیجه گردش کار دچار بن‌بست می‌شود که وضعیتی نامطلوب است.
\شروع{شکل}[t]
\raggedright
\centerimg{two-pending-or-joins}{12cm}
\vspace{0.5em}
\شرح{منتظر بودن دو پیوند از نوع ``یا'' در مثال خرید برخط}
\برچسب{شکل: منتظر بودن دو پیوند از نوع ``یا'' در مثال خرید برخط}
\پایان{شکل}

برای پیدا کردن این وضعیت در گردش کار اظهار زیر را به توصیف کلی اضافه می‌کنیم:



\شروع{شکل}[H]
\raggedright
\includegraphics[width=16cm]{two-pending-or-joins-assertion-in-shopping}
\vspace{0.5em}
\برچسب{شکل: اظهار نوشته‌شده برای بررسی منتظر بودن دو پیوند از نوع ``یا'' در مثال خرید برخط}
\پایان{شکل}

\پایان{فقرات}




\قسمت{ایجاد موارد آزمون و اعمال آن‌ها بر روی مدل}
آزمون نرم افزار یک فرآیند حیاتی است که کیفیت محصولات نرم افزاری را تضمین می کند. در این پژوهش ما آزمون مدل صوری ترجمه‌شده‌ی نرم افزار با معماری میکروسرویس را با استفاده از معیارهای پوشش بند فعال محدود (RACC) پیشنهاد می کنیم.

پس از اطمینان از قابل دسترس بودن همه‌ی وظایف و هم‌چنین اطمینان از عدم برخورد با وضعیت‌های غیرمجاز به آزمون گردش‌ کار می‌پردازیم در رویکرد ارائه‌شده در این پژوهش ما جریان کنترل گردش کار را مورد آزمون قرار می‌دهیم، ما برای آزمون مدل صوری به زبان الوی، موارد آزمون را در مکان‌هایی که در آن‌ها جریان کنترل تعیین می‌شود، تولید می‌کنیم. 

در یک سیستم نرم‌افزاری با معماری میکروسرویس، هر خدمت یا یک مجموعه خدمت توسط یک میکروسرویس انجام می‎گیرد، پاسخ به درخواست کاربر از نقطه‎ای آغاز می‌شود و سپس گردش کار در میان میکروسرویس ها شامل ارتباط بین سرویس های مختلف برای تکمیل یک فراروند است. هر میکروسرویس ورودی را از سرویس های دیگر دریافت می کند، عملکرد خود را انجام می دهد و خروجی را به سرویس های دیگر می فرستد. به عبارتی گردش کار به عملکرد هر سرویس وابسته است و تعیین مسیر در گردش کاری با توجه به ورودی و خروجی‌های سرویس‌ها انجام می‌گیرد. 

مدل‌ گردش کاری میکروسرویس‌ها مدل‌سازی سطح بالایی است و به مدل‌سازی از عملکرد داخلی سرویس‌ها پرداخته نمی‌شود و تنها ارتباط آن‌ها و نحوه‌ی تعیین جریان کنترل در آن مشخص می‌شود. در نتیجه ترجمه‌ی این مدل‌ها نیز به صورت صوری شامل توصیف عملکرد داخلی میکروسرویس‌ها نیست و در آن صرفا به توصیف نحوه‌ی گردش کار و جریان کنترل در سیستم پرداخته می‌شود. جریان کنترل در زبان‌ یاول، به وسیله‌ی پیوند یا انشعاب‌ها از انواع مختلف انجام می‌شود برای این درگاه‌ها بسته به نوع آن‌ها شروطی تعیین می‎شود و تصمیم‌گیری با توجه به این شروط و همچنین نوع درگاه گرفته می‌شود. 
ما برای تولید موارد آزمون مسندهایی که در انشعاب‌ها توسط طراح نوشته می‌شوند را مبنا قرار می‌دهیم. گزاره‌های شرطی موجود در انشعاب‌ها در ترجمه‌ی مدل‌ها به زبان الوی در نظر گرفته می‎شوند و در توصیف وظایف آن‌ها نیز توصیف می‎شوند. با پیش‌پردازش توصیف به دست ‌آمده، انشعاب‌ها را می‌یابیم و سپس مسندهای آن‌ها را برای تولید موارد آزمون استخراج می‌کنیم. 

در روشی که ارائه داده‌ایم ما از RACC برای تولید موارد آزمون استفاده می‌کنیم. برای تولید خودکار موارد آزمون به روش RACC ماژولی ساختیم که با ورودی مسند، جفت‌های موارد آزمون را برای RACC تولید می‌کند. 

برای به دست آوردن زوج موارد آزمون بر اساس RACC، نیاز است که گزاره‌های اصلی که ارزش کلی مسند را تعیین می‌کنند بیابیم. نیازمندی آزمون برای هر ci دو شرط دارد: گزاره‌ی اصلی به ارزش "درست" و به ارزش "نادرست" اطلاق شود. مقادیر انتخاب شده برای گزاره‌های فرعی باید در زمانی که گزاره‌ی اصلی "درست" است با زمانی که گزاره‌ی اصلی  "نادرست" است یکسان باشد\مرجع{alloybook}.
می‌توان با محاسبه‌ی مسند به روش زیر برای یک گزاره‌ی خاص، تعیین ارزش مسند را به ارزش گزاره‌ی اصلی وابسته کرد. 
$P_a = p_{a=True}\oplus p_{a=False}$
برای تولید زوج موارد آزمون بعد از محاسبه‌ی مسند به صورتی که ارزش آن را گزاره‌ی اصلی تعیین کند، ارزش گزاره اصلی را یک بار برابر با "درست" و یک بار نیز برابر "نادرست" قرار می‌دهیم. برای این که بتوانیم ارزش گزاره‌ی اصلی را برابر "درست" یا "نادرست" قرار بدهیم به گونه‌ای متغیر‌های مورد استفاده در گزاره‌ی شرطی را مقداردهی می‌کنیم تا گزاره ارزش مورد نظر را پیدا کند.
در روشی که ما برای تولید موارد آزمون استفاده کردیم، ابتدا با تجزیه و تحلیل مسند موجود در جریان خروجی گزاره‌های شرطی تشکیل‌دهنده‌ی مسند و متغیرهای موجود در آن‌ها را را استخراج می‌کنیم و  سپس درخت عبارت منطقی را می‌سازیم. درخت عبارت منطقی درختی دودویی است که برای نشان دادن عبارات منطقی و استدلال‌های منطقی استفاده می‌شود. برگ‌های این درخت همان گزاره‌های شرطی هستند و راس‌های میانی آن عملگرهای منطقی هستند. مسند زیر را در نظر بگیرید شکل ~\رجوع{شکل: درخت عبارت منطقی برای گزاره‌های مسند نمونه} نمایان‌گر درخت عبارت منطقی آن است.
$(a \wedge b) \vee c$
\شروع{شکل}[t]
\centerimg{expression-diagram}{8cm}
\vspace{0.5em}
\شرح{درخت عبارت منطقی برای گزاره‌های مسند $(a \wedge b) \vee c$}
\برچسب{شکل: درخت عبارت منطقی برای گزاره‌های مسند نمونه}
\پایان{شکل}

با استفاده درخت عبارت منطقی به ازای همه‌ی $P_i$ ها عبارت 
$P_a = p_{a=True}\oplus p_{a=False}$
را محاسبه می‌کنیم، سپس با ثابت نگه‌داشتن ارزش گزاره‌های فرعی ارزش $P_i$ را یک بار برابر "درست" و یک بار برابر "نادرست" قرار می‌دهیم. در انتها با تحلیل گزاره‌ها و تجزیه‌ی آن‌ها به عملگرها و عملوندی تشکیل‌دهنده، مقادیر متغیرها را به گونه‌ای به آن‌ها نسبت می‌دهیم که ارزش مورد نظر برای گزاره حاصل شود.

پس از تولید موارد آزمون، آن‌ها را بر روی توصیف اعمال می‌کنیم. هر یک از موارد آزمون را با زبان الوی به صورت صوری توصیف می‌کنیم، سپس توصیف را به توصیف کلی گردش کار اضافه می‌کنیم و آن را با API های ابزار تحلیل الوی اجرا می‌کنیم. 

اجرای آزمون‌ها با استفاده از تحلیل‌گر الوی انجام می‌شود و در نتیجه در فضایی کوچک و با حالت‌های محدود‌شده انجام می‌گیرد\مرجع{jackson2019}. در نتیجه نمی‌توان موفقیت اجرای یک آزمون را معادل موفقیت همیشگی برنامه با ورودی‎های مورد آزمون دانست، اما ناموفق بودن آزمون وجود خطا در برنامه با وجود ورودی‌های مورد آزمون را تضمین می‌کند. به همین دلیل استفاده از تحلیل‌گر الوی به قدرت روش ارائه‌شده در تشخیص خطاها در برنامه می‎‌افزاید.

زوج‌های موارد آزمون تولید شده برای مجموعه متغیرهای استفاده شده در عبارات شرطی در هر درگاه تولید می‌شوند. طبق تعریف RACC ارزش هر متغیری که مورد آزمون است، تعیین‌کننده‌ی ارزش کلی مسند است. سازگار بودن توصیفی که از اعمال موارد آزمون به دست می‎آید، به معنی موفق بودن آزمون است.

در زیر نمونه‌ای از تولید موارد آزمون برای مثال خرید برخط آمده است. همان‌طور که در شکل زیر می‌بینید خروج از "تاییدکننده‌ی پرداخت" و ورود به "اتمام خرید" شرط زیر را داراست:
\شروع{شکل}[t]
\centerimg{payment-test-example.png}{8cm}
\vspace{0.5em}
\شرح{وظیفه‌ی انتخاب شده‌ی دارای گزاره‌های شرطی برای ایجاد آزمون}
\برچسب{وظیفه‌ی انتخاب شده‌ی دارای گزاره‌های شرطی برای ایجاد آزمون}
\پایان{شکل}

\شروع{شکل}[H]
\raggedright
\includegraphics[width=10cm]{prdicates-in-payment-test}
\vspace{0.5em}
\برچسب{شکل: گزاره‌های موجود در جریان‌های خروجی وظیفه‌ی پرداخت}
\پایان{شکل}
است.

توصیفی که برای اجرای موارد آزمون تولید شده بر اساس معیار پوشش RACC ایجاد می‌شود به صورت زیر است:



\شروع{شکل}[H]
\raggedright
\includegraphics[width=16cm]{test-specification}
\vspace{0.5em}
\برچسب{شکل: گزاره‌های موجود در جریان‌های خروجی وظیفه‌ی پرداخت}
\پایان{شکل}


  

